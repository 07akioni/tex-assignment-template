\documentclass[acmlarge,screen]{acmart}
\usepackage{xeCJK}
\usepackage{multicol}
\usepackage{array, stackengine}
\usepackage{listings}
\usepackage{fontspec}
\usepackage{lstfiracode}

\def\verbatim@nolig@list{}

\setmonofont[
  BoldFont={Fira Code Bold},
  Contextuals=Alternate
]{Fira Code}

\lstset{
	basicstyle=\linespread{1.11}\fontsize{9pt}{9pt}\ttfamily,
	style=FiraCodeStyle,
	keywordstyle=\bfseries,
	breaklines=true,
	% numbers=left
}

\begin{document}
\title{作业 2}
\author{张乐聪 (1801213722)}
\maketitle
\section{问题 4.17}
\subsection{4.17(a)}
\begin{multicols}{2}
Euclid:
\begin{align*}
	gcd(2152, 764) &= gcd(754, 624)\\
	&= gcd(624, 140)\\
	&= gcd(140, 64)\\
	&= gcd(64, 12)\\
	&= gcd(12, 4)\\
	&= gcd(4, 0)\\
	&= 4
\end{align*}
\columnbreak \\
Stein:
\begin{align*}
	A_1&=2152 & B_1&=764 & C_1&=1\\
	A_2&=1076 & B_2&=382 & C_2&=2\\
	A_3&=538 & B_3&=191 & C_3&=4\\
	A_4&=269 & B_4&=191 & C_4&=4\\
	A_5&=78 & B_5&=191 & C_5&=4\\
	A_6&=39 & B_6&=191 & C_6&=4\\
	A_7&=152 & B_7&=39 & C_7&=4\\
	A_8&=76 & B_8&=39 & C_8&=4\\
	A_9&=38 & B_9&=39 & C_9&=4\\
	A_{10}&=1 & B_{10}&=38 & C_{10}&=4\\
	A_{11}&=37 & B_{11}&=1 & C_{11}&=4\\
	A_{12}&=36 & B_{12}&=1 & C_{12}&=4\\
	A_{13}&=18 & B_{13}&=1 & C_{13}&=4\\
	A_{14}&=9 & B_{14}&=1 & C_{14}&=4\\
	A_{15}&=8 & B_{15}&=1 & C_{15}&=4\\
	A_{16}&=4 & B_{16}&=1 & C_{16}&=4\\
	A_{17}&=2 & B_{17}&=1 & C_{17}&=4\\
	A_{18}&=1 & B_{18}&=1 & C_{18}&=4
\end{align*}
since $A_{18}=B_{18}$, $gcd(2152, 764)=A_{18}*C_{18}=4$
\end{multicols}
\subsection{4.17(b)}
和 Euclid 算法相比,Stein 算法使用不使用除法而是使用右移和减法,它使用的指令在 CPU 运算的速度更快。
\section{问题 4.19}
\subsection{4.19(a)}
\begin{table}
  \caption{4.19(a)}
  \label{tab:4.19(a)}
  \begin{tabular}{m{5em} m{5em} m{5em} m{5em} m{5em}}
    \toprule
    $i$&$r_i$&$q_i$&$u_i$&$v_i$\\
    \midrule
    $-1$&$4321$&&$1$&$0$\\
    $0$&$1234$&&$0$&$1$\\
	$1$&$619$&$3$&$1$&$-3$\\
	$2$&$615$&$1$&$-1$&$4$\\
	$3$&$4$&$1$&$2$&$-7$\\
	$4$&$3$&$153$&$-307$&$1075$\\
	$5$&$1$&$1$&$309$&$-1082$\\
	$6$&$0$&&&\\
  \bottomrule
\end{tabular}
\end{table}
通过表格 \ref{tab:4.19(a)} 的$i=5$那一行可知:
\begin{align*}
	309*4321-1082*1234=1\\
	\Rightarrow-1082*1234\equiv1\ (\text{mod}\ 4321)\\
	\Rightarrow3239*1234\equiv1\ (\text{mod}\ 4321)
\end{align*}
$1234$的逆元是$3239$。
\subsection{4.19(b)}
\begin{table}
  \caption{4.19(b)}
  \label{tab:4.19(b)}
  \begin{tabular}{m{5em} m{5em} m{5em} m{5em} m{5em}}
    \toprule
    $i$&$r_i$&$q_i$&$u_i$&$v_i$\\
    \midrule
    $-1$&$40902$&&$1$&$0$\\
    $0$&$24140$&&$0$&$1$\\
	$1$&$16762$&$1$&$1$&$-1$\\
	$2$&$7378$&$1$&$-1$&$2$\\
	$3$&$2006$&$2$&$3$&$-5$\\
	$4$&$1360$&$3$&$-10$&$17$\\
	$5$&$646$&$1$&$13$&$-22$\\
	$6$&$68$&$2$&$-36$&$61$\\
	$7$&$34$&$9$&$337$&$-571$\\
	$8$&$0$&&&\\
  \bottomrule
\end{tabular}
\end{table}
通过表格 \ref{tab:4.19(b)} 的$i=7$那一行可知:
\begin{align*}
	337*40902-571*24140=34\\
	\Rightarrow-571*24140\equiv34\ (\text{mod}\ 40902)\\
	\Rightarrow40331*24140\equiv34\ (\text{mod}\ 40902)
\end{align*}
所以$gcd(40902, 24140)=34$,因为$1$是单位元,$24140$的逆元$x$应该满足
\begin{align*}
24140x\equiv1\ (\text{mod}\ 40902)	
\end{align*}
也就是说$gcd(40902, 24140)=1$,这和$gcd(40902, 24140)=34$是矛盾的。所以$24140$的逆元不存在。
\subsection{4.19(c)}
\begin{table}
  \caption{4.19(c)}
  \label{tab:4.19(c)}
  \begin{tabular}{m{5em} m{5em} m{5em} m{5em} m{5em}}
    \toprule
    $i$&$r_i$&$q_i$&$u_i$&$v_i$\\
    \midrule
    $-1$&$1769$&&$1$&$0$\\
    $0$&$550$&&$0$&$1$\\
	$1$&$119$&$3$&$1$&$-3$\\
	$2$&$74$&$4$&$-4$&$13$\\
	$3$&$45$&$1$&$5$&$-16$\\
	$4$&$29$&$1$&$-9$&$29$\\
	$5$&$16$&$1$&$14$&$-45$\\
	$6$&$13$&$1$&$-23$&$74$\\
	$7$&$3$&$1$&$37$&$-119$\\
	$8$&$1$&$4$&$-171$&$550$\\
	$9$&$0$&&&\\
  \bottomrule
\end{tabular}
\end{table}
通过 表格 \ref{tab:4.19(c)} 的$i=8$那一行可知:
\begin{align*}
	-171*1769+550*550=1\\
	\Rightarrow550*550\equiv1\ (\text{mod}\ 1769)
\end{align*}
所以$550$的逆元是$550$。
\section{问题 4.27}
\begin{table}
  \caption{4.27}
  \label{tab:4.27}
  \begin{tabular}{m{5em} m{5em} m{5em} m{5em} m{5em}}
    \toprule
    $i$&$r_i(x)$&$q_i(x)$&$u_i(x)$&$v_i(x)$\\
    \midrule
    $-1$&$x^4+x+1$&&$1$&$0$\\
    $0$&$x^3+x+1$&&$0$&$1$\\
	$1$&$x^2+1$&$x$&$1$&$-x$\\
	$2$&$1$&$x$&$-x$&$x^2+1$\\
	$3$&$0$&&&\\
  \bottomrule
\end{tabular}
\end{table}
通过 表格 \ref{tab:4.27} 的$i=2$那一行可知:
\begin{align*}
	(x^3+x+1)(x^2+1)-(x^4+x+1)x=1\\
	\Rightarrow(x^3+x+1)(x^2+1)\equiv1\ (\text{mod}\ x^4+x+1)
\end{align*}
所以$x^3+x+1$的逆元是$x^2+1$。
\section{问题 8.1}
\subsection{8.1(a)}
\label{8.1(a)}
不存在。
\begin{align*}
	X &= 1+p_1p_2...p_n\\
	\Rightarrow X &\equiv 1\ (\text{mod}\ p_i)\quad(i=1,2,...,n)\\
	\Rightarrow p_i&\nmid X (i=1,2,...,n)
\end{align*}
所以不存在$P_m$可以整除$X$。
\subsection{8.1(b)}
\label{8.1(b)}
$m\notin \{1, 2, ..., n\}$
\subsection{8.1(c)}
假设只有$n$个质数,通过章节\ref{8.1(a)}和章节\ref{8.1(b)}可知,$X$不能被任何质数整除,所以$X$是一个不在$\{p_1,p_2,...,p_n\}$里面的质数,所以就存在$n+1$个质数,这和只存在$n$个质数矛盾。所以质数的数量是无限的。
\subsection{8.1(d)}
因为$X$是一个大于$p_n$的质数,如果$P_{n+1}$是第一个大于$p_n$的质数,$p_n<P_{n+1} \leq X$。所以
\begin{displaymath}
	P_n\leq 1+p_1p_2...p_n
\end{displaymath}
\section{问题 8.8}
假设$x$和$35$是互质的。
\begin{align*}
	x^{\phi (35)}&\equiv1\ (\text{mod}\ 35)\\
	x^{\phi (7) \phi (5)}&\equiv1\ (\text{mod}\ 35)\\
	x^{6*4}&\equiv1\ (\text{mod}\ 35)\\
	x^{24}&\equiv1\ (\text{mod}\ 35)\\
	\Rightarrow x^{72}&\equiv1\ (\text{mod}\ 35)\\
\end{align*}
假设$x^{85}\equiv6\ (\text{mod}\ 35)$.
\begin{align*}
	\Rightarrow x^{85-72}&\equiv6\ (\text{mod}\ 35)\\
	\Rightarrow x^{13}&\equiv6\ (\text{mod}\ 35)\\
	\Rightarrow x^{26}&\equiv1\ (\text{mod}\ 35)\\
	\Rightarrow x^{26-24}&\equiv1\ (\text{mod}\ 35)\\
	\Rightarrow x^{2}&\equiv1\ (\text{mod}\ 35)\\
	\Rightarrow (x+1)(x-1) &\mid 35\\
	\Rightarrow (x+1)&\mid35\\
	\text{or}\\
	(x-1)&\mid35\\
	\text{or}\\
	(x+1)\mid5,(x-1)&\mid7\\
	\text{or}\\
	(x+1)\mid7,(x-1)&\mid5
\end{align*}
所以$x=34+35k$或$x=1+35k$或$x=99+35k$(使用中国剩余定理)或$x=111+35k$(使用中国剩余定理)。因为$0<x<28$,所以$x=1$或$x=6$,如果$x=1$,$x^{85}\equiv1\nequiv6\ (\text{mod}\ 35)$。经过验证后,$x=6$。
\section{问题 8.12}
假设$m$是一个整数并且$p$是一个质数。我们把所有小小于等于$mp$的数写成下面的形式:
\begin{center}
	\begin{tabular}{m{5em} m{5em} m{5em} m{5em} m{5em}}
    $1$&$2$&$...$&$m-1$&$m$\\
    $m+1$&$m+2$&$...$&$2m-1$&$2m$\\
    $2m+1$&$2m+2$&$...$&$3m-1$&$3m$\\
    $...$&$...$&$...$&$...$&$...$\\
    $(p-1)m+1$&$(p-1)m+2$&$...$&$pm-1$&$pm$\\
\end{tabular}
\end{center}
易知$gcd(x,m)=gcd(x+km, m)$,所以如果一列的第一个数和$m$互质,那么整列的数都和$m$互质,如果一列的第一个数和$m$ 不互质,那么整列的数和$m$不互质,也就是说整列和$mp$不互质。\\\\
我们先证明如果$m$和$p$不互质,那么$\phi(mp)=\phi(m)*p$。因为$m$和$p$不互质,所以如果$x$和$m$互质,$x$和$p$ 也互质,所以$x$和$mp$是互质的,也就是说$gcd(x, m)=1\Rightarrow gcd(x, mp)=1$。所以如果一列的第一个数和$m$互质,整列都和$m$互质,所以整列都和$mp$互质。所以全部小于$pm$并且和$mp$互质的数的数量是$\phi(m)p$。\\\\
然后我们证明如果$m$和$p$互质,$\phi(mp)=\phi(m)*(p-1)$。我们有$\phi(m)$列里面所有的数字都和$m$互质,剩下所有的数和$m$不互质,也就和$mp$不互质。因为$m$和$p$互质,所以每一列至多有一个数和$p$不互质,若不然
\begin{align*}
	gcd(k_1m+b,p)\neq1 \Rightarrow gcd(k_1m+b,p)=p \Rightarrow p\mid k_1m+b\\
	gcd(k_2m+b,p)\neq1 \Rightarrow gcd(k_2m+b,p)=p \Rightarrow p\mid k_2m+b\\
	\Rightarrow p \mid(k_2-k_1)m
\end{align*}
因为$k1-k2<p$,所以$p\mid m$,这和$m$与$p$是矛盾的。\\\\
我们能证明每一列所有的数模$p$互不同余,若不然
\begin{align*}
	k_1m+b&\equiv r\ (\text{mod}\ p)\\
	k_2m+b&\equiv r\ (\text{mod}\ p)\\
	\Rightarrow p \mid(k2-k1)m
\end{align*}
因为$k1-k2<p$,所以$p\mid m$,这和$m$与$p$是矛盾的。每一列所有的数模$p$互不同余,所以每一列至少有一个数和$p$不互质。结合之前的结论,每一列有且只有一个数和$p$不互质。\\\\
所以有$\phi(m)*p-\phi(m)$个数$m$和$p$都互质,因为$gcd(x, m)=1, gcd(x, p) \Rightarrow gcd(x, mp)=1$,所以有$\phi(m)*p-\phi(m)$个数字和$mp$互质,也就是说$\phi(mp)=\phi(m)(p-1)$.\\\\
如果$m=p_1^{\alpha_1}p_2^{\alpha_2}...p_n^{\alpha_n}$,我们可以使用下面的步骤计算$\phi(m)$(通过之前证明的两个结论)。
\begin{align*}
	\phi(p_1)&=p_1-1\\
	\phi(p_1p_2)&=(p_1-1)(p_2-1)\\
	...\\
	\phi(p_1p_2...p_n)&=(p_1-1)(p_2-1)...(p_n-1)\\
	\phi(p_1^2p_2...p_n)&=p_1(p_1-1)(p_2-1)...(p_n-1)\\
	...\\
	\phi(p_1^{\alpha_1}p_2...p_n)&=p_1^{\alpha_1-1}(p_1-1)(p_2-1)...(p_n-1)\\
	...\\
	\phi(p_1^{\alpha_1}p_2^{\alpha_2}...p_n)&=p_1^{\alpha_1-1}(p_1-1)p_2^{\alpha_2-1}(p_2-1)...(p_n-1)\\
	...\\
	\phi(p_1^{\alpha_1}p_2^{\alpha_2}...p_n^{\alpha_n})&=p_1^{\alpha_1-1}(p_1-1)p_2^{\alpha_2-1}(p_2-1)...p_n^{\alpha_n-1}(p_n-1)\\
	\phi(m)=&\prod_{i=1}^{n}p_i^{\alpha_i-1}(p_i-1) 
\end{align*}
\section{问题 8.14}
\subsection{8.14(a)}
$3$,因为$3\mid 2^3-2$。
\subsection{8.14(b)}
因为$n$是不等于$2$的质数,$n$和$2^{n - 1}$是互质的,通过欧拉定理可知
\begin{align*}
	2^{\phi(n)}&\equiv1\ (\text{mod}\ n)\\
	\Rightarrow 2^{n-1}&\equiv1\ (\text{mod}\ n)\\
	\Rightarrow 2^n&\equiv2\ (\text{mod}\ n)\\
	\Rightarrow n&\mid 2^n-2
\end{align*}
\subsection{8.14(c)}
$561=3*11*17$
\begin{align*}
	561&=512+32+16+1\\
	2^{1}&\equiv2\ (\text{mod}\ 561)\\
	2^{2}&\equiv4\ (\text{mod}\ 561)\\
	2^{4}&\equiv16\ (\text{mod}\ 561)\\
	2^{8}&\equiv256\ (\text{mod}\ 561)\\
	2^{16}&\equiv460\ (\text{mod}\ 561)\\
	2^{32}&\equiv103\ (\text{mod}\ 561)\\	
	2^{64}&\equiv511\ (\text{mod}\ 561)\\
	2^{128}&\equiv256\ (\text{mod}\ 561)\\
	2^{256}&\equiv460\ (\text{mod}\ 561)\\
	2^{512}&\equiv103\ (\text{mod}\ 561)\\
	\Rightarrow 2^{561}=2^{512} * 2^{32} + 2^{16} + 2^{1}&\equiv103 * 103 * 460 * 2\equiv2\ (\text{mod}\ 561)\\
	\Rightarrow 561 &\mid 2^{561} - 2
\end{align*}
\subsection{8.14(d)}
通过欧拉定理可知
\begin{align*}
	2^{\phi(341)}&\equiv1\ (\text{mod}\ 341)\\
	2^{\phi(11) * \phi(31)}&\equiv1\ (\text{mod}\ 341)\\
	2^{10*30}&\equiv1\ (\text{mod}\ 341)\\
	2^{300}&\equiv1\ (\text{mod}\ 341)\\
\end{align*}
又
\begin{align*}
	2^{40}-1&=(2^{20}+1)(2^{20}-1)\\
	&=(2^{20}+1)(2^{10}+1)(2^{10}-1)\\
	&=(2^{20}+1)(2^{10}+1)(2^{5}+1)(2^{5}-1)\\
	&=(2^{20}+1)(2^{10}+1)*33*31\\
	&=(2^{20}+1)(2^{10}+1)*3*11*31\\
	\Rightarrow 2^{40}-1 &\mid 11,  2^{40}-1 \mid 31\\
	\Rightarrow 2^{40}-1 &\mid 341\\
	\Rightarrow 2^{40}&\equiv1\ (\text{mod}\ 341)\\
\end{align*}
所以
\begin{align*}
2^{340}&\equiv1\ (\text{mod}\ 341)\\
2^{341}&\equiv2\ (\text{mod}\ 341)\\	
\end{align*}

\section{问题 8.19}
第$371$天,假设是第$x$天。我们有下列同余方程组,从里面选出四个:
\begin{multicols}{2}
\begin{align*} 
	x&\equiv1\ (\text{mod}\ 2)\\
	x&\equiv2\ (\text{mod}\ 3)\\
	x&\equiv3\ (\text{mod}\ 4)\\
	x&\equiv4\ (\text{mod}\ 1)\\
	x&\equiv5\ (\text{mod}\ 6)\\
	x&\equiv6\ (\text{mod}\ 5)\\
	x&\equiv0\ (\text{mod}\ 7)\\
\end{align*}
\columnbreak \\
\begin{align*}
	x&\equiv1\ (\text{mod}\ 2)\\
	x&\equiv2\ (\text{mod}\ 3)\\
	x&\equiv1\ (\text{mod}\ 5)\\
	x&\equiv0\ (\text{mod}\ 7)
\end{align*}
\end{multicols}
\noindent 我们先来解右侧的四个方程,通过中国剩余定理可知
\begin{multicols}{2}
\begin{align*}
	M&=m_1m_2m_3=2*3*5*7=210,\\
	a_1&=1,a_2=2,a_3=1,a_4=0\\
	x&=a_1t_1M_1+a_2t_2M_2+a_3t_3M_3+a_4t_4M_4+KM\\
	x&=t_1M/m_1+2t_2M/m_2+t_3M/m_3\\
	x&=105t_1+140t_2+42t_3+210K\quad(K \in \mathbb{Z})
\end{align*}
\columnbreak\\
\begin{align*}
	105t_1&\equiv1\ (\text{mod}\ 2)\Rightarrow t_1\equiv1\ (\text{mod}\ 2)\\
	70t_2&\equiv1\ (\text{mod}\ 3)\Rightarrow t_2\equiv1\ (\text{mod}\ 3)\\
	42t_3&\equiv1\ (\text{mod}\ 5)\Rightarrow t_3\equiv3\ (\text{mod}\ 5)
\end{align*}
\end{multicols}
\noindent 所以
\begin{align*}
	x&=371+210K\quad(K \in \mathbb{Z})
\end{align*}
$x$ 已经满足 $x\equiv4\ (\text{mod}\ 1)$。所以还需要$x\equiv3\ (\text{mod}\ 4)$和$x\equiv5\ (\text{mod}\ 6)$,因此
\begin{align*}
4\mid 210K+371-3=210K+368\\
6\mid 210K+371-5=210K+366
\end{align*}
因为$6\mid 210K+366$总是成立,$4\mid210K+368 \Rightarrow 2\mid105K$,所以$K$是个偶数,所以最小的$x>0$是$371$.
\section{问题 8.22}
\begin{lstlisting}[language=C++]
int fastPow(int base, int exp, int mod) {
    if (base == 1) return 1;
    if (exp == 0) return 1;
    bool start = false;
    int ret = base % mod;
    for (int i = 30; i >= 0; --i) {
        int mask = 1 << i;
        if (start) {
            int square = (((ret % mod) * (ret % mod)) % mod);
            if (mask & exp) {
                ret = (square * (base % mod)) % mod;
            } else {
                ret = square;
            }
        } else if (mask & exp) {
            start = true;
        }
    }
    return ret;
}
\end{lstlisting}
\section{问题 8.23}
\begin{lstlisting}[language=C++]
int gcd (int a, int b) {
    if (b == 0) return a;
    return gcd(b, a % b);
}

bool witnessTest(int n, int witness) {
    if (fastPow(witness, n - 1, n) == 1) return true;
    else return false;
}

// Miller Rabin Test for k times
bool isPrime(int n, int k) {
    if (n < 2) return false;
    int counter = k;
    while (counter) {
        int witness = random() % 10007;
        if (gcd(witness, n) == 1) {
            if (!witnessTest(n, witness)) return false;
            counter--;
        }
    }
    return true;
}
\end{lstlisting}
\end{document}
